\documentclass{article}
\usepackage{graphicx}
\usepackage[a4paper, margin=3cm]{geometry}
\usepackage{fancyhdr}
\usepackage{lastpage}
\usepackage{hyperref}
\usepackage{polski}
\usepackage{amsmath}
\usepackage{blindtext}
\usepackage{multicol}
\usepackage{listings}

\title{Wprowadzenie do cyberbezpieczeństwa\\Temat 1:\\Uwierzytelnianie klienta SSH\\za pomocą kluczy prywatnych}
\author{Jakub Stachowicz, 198302\\Jan Wiśniewski, 197662}
\date{2 czerwca 2025}

\pagestyle{fancy}
\fancyhf{} % Clear header and footer
\fancyfoot[C]{Strona \thepage\ z \pageref{LastPage}}

\renewcommand{\headrulewidth}{0pt} % Remove the horizontal bar at the top
\renewcommand{\footrulewidth}{0pt} % Remove the horizontal bar at the bottom if needed

\renewcommand{\contentsname}{Spis rzeczy} % Change the TOC title

\newcounter{firstbib} % New couter for bibliografy

\begin{document}
\maketitle
\newpage
\tableofcontents
\newpage


\begin{multicols}{2}
[
\section{Omówienie SSH i konfiguracja serwera}
]
\subsection{Czym jest SSH?}
SSH (Secure Shell) to protokół sieciowy, który służy do zdalnego logowania się do innego komputera (serwera) w sposób bezpieczny. Umożliwia zarządzanie systemem operacyjnym, przesyłanie plików oraz wykonywanie poleceń na odległość -- wszystko to przy użyciu szyfrowanego połączenia\cite[The SSH protocol]{whatisssh}.

Uwierzytelnianie w SSH może odbywać się na dwa sposoby: za pomocą hasła lub kluczy kryptograficznych. Logowanie hasłem jest proste, ale mniej bezpieczne, ponieważ narażone jest na ataki typu brute-force. Znacznie bezpieczniejszą i częściej stosowaną metodą jest logowanie przy użyciu kluczy SSH, które opiera się na parze klucz prywatny–publiczny. Klucz publiczny umieszczany jest na serwerze, a prywatny pozostaje na komputerze użytkownika, dzięki czemu możliwe jest uwierzytelnienie bez podawania hasła. Metoda ta jest szczególnie przydatna w automatyzacji i pracy z wieloma serwerami\cite[Automate with SSH keys, but manage them]{whatisssh}.

\subsection{Algorytmy kryptograficzne w SSH}
Temat algorytmów kryptograficznych w protokole SSH dotyczy zarówno szyfrowania samego ruchu sieciowego między klientem a serwerem, jak i logowania i autoryzacji. Niniejsze opracowanie skupia się na algorytmach stosowanych w autoryzacji za pomocą kluczy prywatnych.

W protokole SSH można używać kilku algorytmów generujących parę kluczy, w standardzie zdefiniowano RSA, DSS (DSA) oraz umożliwiono definiowanie innych kluczy\cite[6.6.~s.~13]{rfc4253}. Najpopularniejsze to RSA, ECDSA i Ed25519 (a dawniej używano też DSA, które dziś jest uważane za przestarzałe). OpenSSH wspiera następujące typy kluczy: DSA, RSA, ECDSA oraz Ed25519\cite[User key generation]{learnms}. Obecnie zaleca się wybór silniejszych algorytmów, np. RSA lub Ed25519 -- OpenSSH 7.0 i nowsze domyślnie wyłączają słaby DSA, a od wersji 10.0 wsparcie dla tego algorytmu zostało usunięte w całości\cite{openssh10}.

Dalsza część artukułu skupia się na użyciu kluczy RSA, jednak pozostałych algorytmów można użyć w bardzo podobny sposób. Najczęściej -- poprzez zmianę parametru \verb|rsa| na np. \verb|ed25519|.

\subsection{Konfiguracja z wymuszonym użyciem kluczy}
Aby serwer SSH (usługa \verb|sshd|) zezwalał tylko na logowanie kluczem i wyłączał logowanie hasłem, należy zmodyfikować plik konfiguracyjny znajdujący się najczęściej pod ścieżką \verb|/etc/ssh/sshd_config|\cite{sshconfig}. Ważne dyrektywy to m.in.:
\begin{enumerate}
    \item \verb|PubkeyAuthentication yes| -- włącza uwierzytelnianie kluczem publicznym (domyślnie zazwyczaj jest włączone).
    
    \item \verb|PasswordAuthentication no| -- wyłącza logowanie hasłem. Po jego ustawieniu serwer nie akceptuje haseł SSH.

    \item \verb|ChallengeResponseAuthentication no| -- wyłącza metody uwierzytelniania oparte na wyzwaniach i odpowiedziach (challenge-response authentication).

    \item \verb|AuthorizedKeysFile| -- określa lokalizację pliku z kluczami publicznymi (domyślnie dla parametru \verb|%h/.ssh/authorized_keys| będzie to \verb|~/.ssh/authorized_keys|).
\end{enumerate}

Po edycji pliku \verb|sshd_config| należy zapisać zmiany i ponownie uruchomić serwis SSH, np.: \verb|sudo systemctl restart sshd| lub \verb|sudo service sshd reload|. Spowoduje to zastosowanie nowych ustawień. 

\end{multicols}

\begin{multicols}{2}
[
\section{Klucze -- generowanie i instalacja}
]
\subsection{Generowanie kluczy}
Parę kluczy (klucz prywatny i publiczny) generuje użytkownik. Klucz prywatny przechowywany jest po stronie klienta, natomiast klucz publiczny -- po stronie serwera.

\subsubsection{W systemie Linux -- ssh-keygen}
W Linuksie (oraz systemach UNIX/macOS) zwykle używa się polecenia \verb|ssh-keygen|. Przykładowo:
\verb|ssh-keygen -t rsa -b 4096|. Polecenie \verb|-t rsa -b 4096| tworzy klucz RSA o długości 4096 bitów. 

Po uruchomieniu program zapyta o ścieżkę do pliku (domyślnie \verb|~/.ssh/id_rsa| dla RSA lub \verb|~/.ssh/id_ed25519| dla Ed25519) i opcjonalne hasło (passphrase). Gdy zakończy generowanie, w katalogu \verb|~/.ssh/| powstaną dwa pliki: \verb|id_rsa| (klucz prywatny) i \verb|id_rsa.pub| (klucz publiczny).

\subsubsection{W systemie Windows -- PuTTYgen}
W systemie Windows często korzysta się z PuTTYgen (narzędzie wchodzące w skład pakietu PuTTY). Uruchamiamy PuTTYgen, wybieramy typ klucza (np. RSA lub Ed25519) i długość (np. 2048 bitów), a następnie klikamy przycisk Generate. PuTTYgen poprosi nas o losowe ruchy myszką w obrębie okna -- to sposób na zebranie entropii do generacji klucza. Gdy pasek postępu dojdzie do końca, w oknie PuTTYgen pojawi się wygenerowany klucz publiczny (tekst zaczynający się np. \verb|ssh-rsa AAAA...|). 

Następnie należy wpisać (dwukrotnie) passphrase chroniący klucz prywatny (zalecane) i zapisać pliki: kliknąć Save public key (np. \verb|mykey.pub|) oraz Save private key (plik \verb|.ppk| dla PuTTY, np. \verb|mykey.ppk|).

W PuTTYgenie można też przekonwertować prywatny klucz do formatu OpenSSH (menu Conversions →
Export OpenSSH key) -- przydaje się to, gdy chcemy używać tego samego klucza w programach innych niż
PuTTY. Plik \verb|.ppk| pozostaje natywnym formatem PuTTY. 

\subsection{Instalacja kluczy}
Po wygenerowaniu kluczy należy je zainstalować -- zarówno po stronie klienta, jak i serwera.

\subsubsection{Klient SSH Linux}
Aby klient logujący się z Linuksa mógł użyć swojego klucza, klucz publiczny klienta należy dodać do pliku
\verb|~/.ssh/authorized_keys| konta docelowego na serwerze. Najprościej to zrobić poleceniem \verb|sshcopy-id| -- narzędzie automatycznie kopiuje nasz klucz publiczny do odpowiedniego pliku na serwerze. Przykład: \verb|ssh-copy-id user@serwer|.

Powoduje to dodanie zawartości klucza \verb|~/.ssh/id_rsa.pub| (lub innego domyślnego klucza) do \verb|~/.ssh/authorized_keys| na serwerze. Jeśli narzędzie nie jest dostępne, można wykonać ręcznie: najpierw zalogować się hasłem, a potem na serwerze stworzyć katalog \verb|.ssh| i dopisać klucz, np.: 
\begin{verbatim}
mkdir -p ~/.ssh
echo "$(cat ~/.ssh/id_rsa.pub)"
            >> ~/.ssh/authorized_keys
chmod 700 ~/.ssh
chmod 600 ~/.ssh/authorized_keys
\end{verbatim}

Następnie należy zadbać o instalację klucza prywatnego po stronie klienta. W tym celu po wygenerowaniu pary kluczy, plik z kluczem prywatnym (\verb|id_rsa|) powinien zostać skopiowany lub przeniesiony tylko do katalogu domowego użytkownika klienta, w podkatalogu \verb|.ssh|: \verb|mv /path/to/id_rsa ~/.ssh/id_rsa|.

Po dodaniu klucza można przetestować logowanie: \verb|ssh user@serwer| już nie powinno prosić o hasło
(chyba że nadaliśmy passphrase do klucza).

\subsubsection{Klient SSH Windows (OpenSSH)}
Windows (np. Windows 10) posiada wbudowany serwer OpenSSH (lub Win32-OpenSSH). Konfiguracja kluczy jest analogiczna: klucz publiczny wrzucamy do pliku \verb|authorized_keys|. Dla konta użytkownika domyślnie jest to ścieżka: \verb|C:\Users\<user>\.ssh\authorized_keys|.

Dla kont administratora przewidziano specjalny plik w katalogu \verb|C:\ProgramData\ssh\|: \verb|administrators_authorized_keys|. Zawartość tych plików można wprowadzić ręcznie np. przez \verb|scp| lub nawet komendami PowerShell (przykład w dokumentacji Microsoft pokazuje użycie ssh z parametrem, który na zdalnym serwerze tworzy katalog \verb|.ssh| i dopisuje klucz do \verb|authorized_keys|). Ważne jest, aby katalog \verb|.ssh| miał ograniczone prawa (tylko właściciel czy administrator), inaczej OpenSSH może odrzucić klucze.

W tym przypadku plik z kluczem prywatnym powinien trafić analogicznie jak na Linuxie do folderu \verb|.ssh| w foledrze domowym użytkownika: \verb|C:\Users\<user>\.ssh\id_rsa|.


Po umieszczeniu klucza publicznego, logowanie SSH przebiega z użyciem klucza, bez hasła.

\subsubsection{Klient SSH PuTTY}
W przypadku PuTTY klucz prywatny musi być w formacie PuTTY (\verb|.ppk|). Jeżeli mamy klucz w formacie OpenSSH (np. wygenerowany wcześniej ssh-keygenem), wystarczy go załadować w PuTTYgen (Load private key) i zapisać jako \verb|.ppk| (Save private key). Następnie, aby PuTTY użył klucza, w oknie konfiguracji sesji PuTTY przechodzimy do Connection → SSH → Auth i w polu ,,Private key file for authentication'' wybieramy nasz plik \verb|.ppk|.

W razie potrzeby możemy też użyć Pageanta -- agenta kluczy PuTTY -- i dodać tam klucz. Klucz publiczny PuTTY (tekst z pola ,,Public key for pasting...'' w PuTTYgen) należy skopiować na serwer do \verb|~/.ssh/authorized_keys| tak samo, jak w poprzednich metodach. Po dodaniu klucza do \verb|authorized_keys| należy skonfigurować PuTTY wskazując wygenerowany prywatny plik \verb|.ppk| i przetestować połączenie.

\end{multicols}

\begin{multicols}{2}
[
\section{Zastosowania SSH}
]

\end{multicols}

\section{Bibliografia}
\renewcommand{\refname}{Artykuły}
\begin{thebibliography}{9}

\bibitem{whatisssh}
    Tatu Ylonen,
    \emph{What is SSH (Secure Shell)?},
    \url{https://www.ssh.com/academy/ssh},
    [dostęp 24.05.2025].
    
\bibitem{learnms}
    Microsoft Learn,
    \emph{Key-based authentication in OpenSSH for Windows},
    \url{https://learn.microsoft.com/en-us/windows-server/administration/openssh/openssh_keymanagement}
    [dostęp 24.05.2025].

\bibitem{openssh10}
    OpenSSH Release Notes,
    \emph{OpenSSH 10.0},
    \url{https://www.openssh.com/txt/release-10.0},
    [dostęp 24.05.2025].
    
\setcounter{firstbib}{\value{enumiv}}
\end{thebibliography}
\renewcommand{\refname}{Dokumentacje techniczne}
\begin{thebibliography}{9}
\setcounter{enumiv}{\value{firstbib}}

\bibitem{rfc4253}
    Chris Lonvick, Tatu Ylonen,
    Styczeń 2006,
    \emph{The Secure Shell (SSH) Transport Layer Protocol},
    \url{https://www.rfc-editor.org/rfc/rfc4253},
    [dostęp 24.05.2025].    

\bibitem{sshconfig}
    OpenBSD manual page server,
    \emph{ssh\_config},
    \url{https://man.openbsd.org/ssh_config},
    [dostęp 24.05.2025].    
    
\end{thebibliography}

\end{document}
